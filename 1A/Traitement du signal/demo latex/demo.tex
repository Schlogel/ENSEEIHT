\documentclass{article}
\usepackage[frenchb]{babel}
\usepackage[T1]{fontenc}
\usepackage{times}

\begin{document}
Calcule théorique de la densité spectrale de puissance du signal modulé en fréquence
x(t) \\[2ex]
Notre signal modulé en fréquence à pour expression:\\[4ex]
$x(t)=(1-NRZ(t))\times cos(2 \pi F_0 t + \phi _0) + NRZ(t)\times cos(2 \pi F_1 t + \phi _1)\\[4ex]$
Or :\\[4ex]
$
x(t)=(1-NRZ(t))\times cos(2 \pi F_0 t + \phi _0) + NRZ(t)\times cos(2 \pi F_1 t + \phi _1)\\[2ex]
=cos(2 \pi F_0 t + \phi _0) - NRZ(t)\times cos(2 \pi F_0 t + \phi _0) + NRZ(t)\times cos(2 \pi F_1 t + \phi _1)\\[4ex]$
On constate qu'il s'agit d'un moment d'ordre 2 donc nous pouvons déduire la fonction d'autocorrélation de x :\\[4ex]
$R_x(t)=R_{NRZ}(\tau)\times R_{cos(2\pi F_0t+\phi _0)}(\tau)+R_{NRZ}(\tau)\times R_{cos(2\pi F_1t+\phi _1)}(\tau)\\[2ex]
=R_{NRZ}(\tau)\times  \frac{1}{2}cos(2\pi F_0t) + R_{NRZ}(\tau)\times  \frac{1}{2}cos(2\pi F_1t)\\[4ex]$
Nous en déduisons la densité spectrale de puissance de x :\\[4ex]
$S_x(f)= \frac{1}{2} \times SNRZ(f)\ast  TF(cos(2\pi F_0\tau))+ \frac{1}{2} \times SNRZ(f)\ast  TF(cos(2\pi F_1\tau))\\[2ex]
= \frac{1}{2} \times SNRZ(f)\ast(\frac{1}{2}(\delta (f-F_0)+\delta (f+F_0))) + \frac{1}{2} \times SNRZ(f)\ast(\frac{1}{2}(\delta (f-F_1)+\delta (f+f_1)))\\[2ex]
= \frac{1}{4}\times SNRZ(f)\ast(\delta (f-F_0)+\delta (f+F_0)) +\frac{1}{4}\times SNRZ(f)\ast(\delta (f-F_1)+\delta (f+F_1))\\[2ex]
= \frac{1}{4}\times (SNRZ(f-F_0)+SNRZ(f+F_0)) +\frac{1}{4}\times (SNRZ(f-F_1)+SNRZ(f+F_1))\\[2ex]
= \frac{1}{4}(SNRZ(f-F_0)+SNRZ(f+F_0)+SNRZ(f-F_1)+SNRZ(f+F_1))$\\[4ex]
Donc :
$S_x(f)\propto SNRZ(f-F_0)+SNRZ(f+F_0)+SNRZ(f-F_1)+SNRZ(f+F_1)$ 


\end{document}